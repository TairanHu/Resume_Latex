
%% Copyright 2006-2013 Xavier Danaux (xdanaux@gmail.com).
%
% This work may be distributed and/or modified under the
% conditions of the LaTeX Project Public License version 1.3c,



\documentclass[11pt,a4paper]{moderncv}   % possible options include font size ('10pt', '11pt' and '12pt'), paper size ('a4paper', 'letterpaper', 'a5paper', 'legalpaper', 'executivepaper' and 'landscape') and font family ('sans' and 'roman')

% moderncv 主题
\moderncvstyle{casual}                        % 选项参数是 ‘casual’, ‘classic’, ‘oldstyle’ 和 ’banking’
%\moderncvstyle{banking}
\moderncvcolor{blue}                          % 选项参数是 ‘blue’ (默认)、‘orange’、‘green’、‘red’、‘purple’ 和 ‘grey’
%\nopagenumbers{}                             % 消除注释以取消自动页码生成功能
\usepackage{amsmath}

% 字符编码
\usepackage[utf8]{inputenc}                   % 替换你正在使用的编码
\usepackage{CJKutf8}

% 调整页面出血
\usepackage[scale=0.86]{geometry}
%\setlength{\hintscolumnwidth}{3cm}           % 如果你希望改变日期栏的宽度

% 个人信息
\name{胡泰然}{}
%\photo[72pt][0.4pt]{pic}
\title{求职意向:后端工程师}
\address{籍贯:湖北黄冈}{性别:男 年龄:23岁}{}
\mobile {\underline{+86 1300 -719 -9462}}
\email{wit\_hutairan@163.com}

%----------------------------------------------------------------------------------
%            内容
%----------------------------------------------------------------------------------
\begin{document}
\begin{CJK}{UTF8}{gbsn}
\makecvtitle

%\section{个人信息}
%\cvdoubleitem{性\quad 别:}{男}{联系电话:}{13007199462}
%\cvdoubleitem{籍\quad 贯:}{湖北黄冈}{出生日期:}{1993年3月1日}
%\cvdoubleitem{专\quad 业:}{高分子材料与工程}{毕业学校:}{武汉工程大学}
%\cvdoubleitem{邮\quad 箱:}{wit\_hutairan@163.com}{人才类型:}{应届毕业生}

\section{教育背景}
\cventry{2011--2015}{工学本科(高分子材料与工程)}{ 武汉工程大学}{}{}{}
\cventry{2015 至今}{工学硕士(机械电子工程)}{ 武汉工程大学}{}{}{}

\section{实践及项目经历}
\cventry{2013.07}{“飞思卡尔”杯智能汽车竞赛}{ 华南赛区二等奖}{ 组长}{}{担任摄像头 B 组组长,参加“飞思卡尔”杯智能汽车竞赛,承担软件编写调试工作}
\cventry{2013.09}{垃圾自动回收系统的制作}{}{}{}{参与垃圾自动回收系统的制作,承担基于 STM32 下位机程序的编写}
\cventry{2013.11}{“电工杯”数学建模比赛}{ 全国二等奖}{ 组长}{}{参加”电工杯“数学建模比赛,承担程序编写和 50\% 模型建立工作}
\cventry{2014.07}{“博创杯”嵌入式物联网设计大赛智能车单项赛}{ 团体全国二等奖、个人全国三等奖}{ 队长}{}{担任领队,参加“博创杯”嵌入式物联网设计大赛智能车单项赛,承担程序编写和队员的指导工作}
\cventry{2015.06}{“挑战杯”比赛}{ 湖北省特等奖、全国三等奖}{ 队员}{}{参加“挑战杯”省赛,独立完成全部硬件端电路的设计及嵌入式程序的编写,负责采集所有数据发送给服务器,并于11月代表学校参加“挑战杯”国赛}
\cventry{2015.11}{远程车载定位系统}{}{}{}{参与远程车载定位系统的开发,负责 STM32 嵌入式程序的编写}

%\cventry{2013.07}{组长}{智能车团队摄像头B组}{武汉工程大学智能车团队}{}{担任摄像头B组组长,参加”飞思卡尔“杯智能汽车竞赛。在为期7个月的比赛中承担摄像头B组的软件编写调试工作,对图像处理算法,小车路径处理得到了自己的见解,并总结出一套自己的调试处理方案,提高了比赛成绩。}
%\cventry{2013.09}{项目参与人}{华中科技大学}{武汉}{}{参与垃圾自动回收系统的制作,承担基于STM32下位机程序的编写。和华中科技大学的研究生一起做这一个项目,在实施过程中学习图像识别技术,并对下位机程序编写有了更深入的认识,使系统更加稳定并得到了他们的好评。}
%\cventry{2013.11}{组长}{数学建模组}{武汉工程大学}{}{参加”电工杯“数学建模比赛,承担程序编写和50\%模型建立工作。前期组织小组成员在备赛的过程中,协调小组成员积极准备,经过2个月的配合,默契程度得到很大提高,在比赛时能一起很好地处理并解决问题锻炼了领导和团队配合能力。}
%\cventry{2014.07}{队长}{武汉工程大学代表队}{武汉工程大学}{}{担任领队,参加”博创杯“嵌入式物联网设计大赛智能车单项赛,承担程序编写和队员的指导工作。在为期4个月的准备中,将前一年的比赛经验和调试处理方案在应用到这个比赛中,并对图像处理及边沿寻找以及路线拟合和去噪滤波算法进行了改进和创新,指导团队其他组别的软件调试,让团队获得更好的成绩。}



\section{奖项及证书}
\cventry{2013.07}{全国大学生“飞思卡尔杯”智能汽车竞赛—华南赛区二等奖}{}{}{}{}
\cventry{2013.12}{“电工杯”数学建模全国二等奖}{}{}{}{}
\cventry{2014.07}{第十届“博创杯”嵌入式物联网设计大赛智能车单项赛团体全国二等奖}{}{}{}{}
\cventry{2014.07}{第十届“博创杯”嵌入式物联网设计大赛智能车单项赛个人全国三等奖}{}{}{}{}
\cventry{2014.11}{软件著作权:APP 应用 CIRCLE 智慧图书管理系统}{}{}{}{}
\cventry{2015.06}{第十届“挑战杯$\cdot$青春在沃”大学生课外学术科技作品竞赛湖北省特等奖}{}{}{}{}
\cventry{2015.06}{实用新型专利:基于 X-bee 的无线传输装置}{}{}{}{}
\cventry{2015.10}{第十四届“挑战杯”中航工业全国大学生课外学术科技作品竞赛全国三等奖}{}{}{}{}
\cventry{2015.10}{ 2016 年“创青春”湖北省大学生创业大赛铜奖}{}{}{}{}

\section{能力及特长}
%\cvlistitem{熟练使用多种关系型数据库}
%\cvlistitem{熟练掌握 C, C\raisebox{0.2ex}{$++$}}
%\cvlistitem{熟练使用 Linux 操作系统}
%\cvlistitem{熟练使用 Git 等版本管理工具}
%\cvlistitem{熟练使用 Latex, Markdown 等语言编写文档}

{\begin{description}
\item[\quad \quad *] 熟练使用多种关系型数据库
\item[\quad \quad *] 熟练掌握 C, C\raisebox{0.2ex}{$++$}
\item[\quad \quad *] 熟练使用 Linux 操作系统
\item[\quad \quad *] 熟练使用 Git 等版本管理工具
\item[\quad \quad *] 熟练使用 Latex, Markdown 等语言编写文档
\end{description}}

\section{GitHub}
%\cvlistitem{\url{https://github.com/TairanHu}}

\quad \quad  \url{https://github.com/TairanHu}

\clearpage\end{CJK}
\end{document}
